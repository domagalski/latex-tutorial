% This document, as well as any compiled output document, is releasted under the
% Creative Commons Attribution Share Alike license. For more information, see
% https://creativecommons.org/licenses/by-sa/3.0/

% Author: Rachel Domagalski
% Email: rsdomagalski@gmail.com
% This document is aimed both at people completely new to latex. It can also
% serve as a reference of examples.

% Document class is required for any and all latex documents. It is the basic
% layout/format of the document you are typing. In this example, the document
% class is an article, and the square brackets are setting the options for font
% and paper size.
\documentclass[11pt,letterpaper]{article}

% In order to use certain features/options, the package for them has to be
% loaded. In general, the syntax for this is \usepackage[options]{packages}. The
% options aren't always necessary, as can be seen from certain \usepackage
% commands.
\usepackage[utf8]{inputenc} % Reads input file as UTF-8.
\usepackage[margin=1in]{geometry} % This sets page layout.
% Packages starting with ams are American Mathematical Society packages. These
% are necessary for many of the math stuff that LaTeX is famous for. It's a good
% idea to always have these 4 ams packages in your preamble for writing any
% scientific paper.
\usepackage{amsmath}
\usepackage{amsfonts}
\usepackage{amssymb}
\usepackage{amsthm}
% The verbatim package is used for sections of unformatted text sections. An
% example application for this would be for including code snippets in your
% paper (actually, I would use something else to get colored syntax).
\usepackage{verbatim}
% The graphicx package is absolutely required if you want to include plots and
% figures in your paper.
\usepackage{graphicx}
% I think this package is for putting captions on the side of figures. This
% package can be ommitted if you choose, but I might use an example of this in
% my sample paper.
\usepackage{sidecap}

% Normally, latex inserts a tab for new paragraphs. In order to disable this,
% the following option is set. It is also worth noting that if this option is
% set, an extra line break (\\) will have to be added to the end of each
% paragraphs. To get regular paragraphs, comment this option out.
\setlength{\parindent}{0pt}

% Define commands that will be widely used.
\newcommand{\br}{\ \\} % Add line breaks
\newcommand{\tab}{\hspace*{2em}} % Add tabs

% This is the title section. I'm only going to use the title, author, and date
% options. For other title options, RTFM.
\title{LaTeX tutorial}
\author{Rachel Domagalski}
% If you leave the date blank, it won't display a date. If you don't include the
% date option, it will include the date of compilation (e.g. today).
\date{August 6, 2013}

% Now we are done with the preamble and we are beginning the actual document.
% This is also a segway to talk about how to define certain sections of the text
% to be formatted in certain ways. The general syntax for formatted sections
% (more commonly known as environments) is this:
% \begin{environment}
% ...
% \end{envrionment}
% The begin-end syntax is basically a declaration of a certain environment. We
% are about to enter the document environment. I will be using less comments
% here as what I have to say can be used as examples of text and formatting.
\begin{document}
\maketitle % This statement is required if you want your title to be displayed.

% To get an abstract in your paper, use the abstract command as such:
\begin{abstract}
    This is an abstract. It should be about a paragraph long and should
    summarize the results of the paper. I'm only going to use a few sentences in
    this example, but it should be enough to get the point across.
\end{abstract}

% Now, I am going to define sections and subsections. The syntax for this does
% not use the begin-end syntax like a lot of the suff in latex. For similar
% constructs, see the texmaker menu items including section and subsection.
\section{Introduction}
% To omit section numbering, add an asterisk as such:
% \section*{Section name}

This is an introduction.
This sentence will be on the same line as the previous one.
This sentence will contain an extra line break to signal the end of a
paragraph.\\

This is a second paragraph. I'm not going to write that much here, as it is for
illustration purposes only. Anyway, I'll show you the subsection construct next.

\subsection{Subsection example}

The double backslash delimiter adds an extra line break to the end of a line.\\
This is a new line, but still technically in the same paragraph as the previous
one due to there not being two line breaks in the text editor after the first
line.

This is going to be in a new paragraph. However, since I didn't add an extra
line end at the end of the previous paragraph, it will look like a new line in
the same paragraph.\\








Adding extra line breaks in the editor will not add extra line breaks in your
document.

\subsection{Lists}

Lists can be made with the itemize or enumerate environments. These lists can be
nested. Here is an example of an itemized list. There shouldn't be much to
comment about in this case, as the syntax is both limited and straightforward.

\begin{itemize}
    \item First item\dots % vim-latexsuite automatically sets ... to \dots
    % vim-latexsuite even does the \dots thing in comments!
    % I have alias emacs='vim' in my .zshrc, because I CAN!
    \item Second item\dots
    \item Third item\dots
    \item Fourth item\dots
    \item etc\dots
\end{itemize}

Here is an example of an enumerated list. It's got basically all of the same
syntax as the first list, but just uses a different environment.

\begin{enumerate}
    \item Pancakes
    \item Waffles
    \item French toast
    \item Bacon
    \item Hash browns
\end{enumerate}

It's also possible to make lists where you choose the labels. To do this, use
the description environment as such.

\begin{description}
% I think that hfill is for horizontal alignment stuff.
\item[First] \hfill \\
    This is the first item.
\item[Second] \hfill \\
    This is the second item.
\item[Third] \hfill \\
    This is the third item.
\end{description}

It's not required to have a line break or anything after each customized list
beginning. Here is another way to do the same list.

\begin{description}
    \item[First] This is the first item.
    \item[Second] This is the second item.
    \item[Third] This is the third item.
\end{description}

\section{Citations}
% See the math typsetting section for examples of citations. I have cited the
% schroedinger equation, the Biot-Savart law, and the Lorentz transformations.
% The command for citing is \cite{referencename}. Reference names will be found
% in an external file that I will describe in the next visible paragraph.

Citing your sources is important, so I will talk about it now and include
citations throughout the rest of the tutorial. Basically, what to do is to make
a list of references in an external file with a .bib extension. The one I am
using for this tutorial will be called reference\_list.bib. I probably won't
comment it much, but it has an easy to follow and logical structure. Now, the
addition of this extra file adds an extra step to the compilation process. When
compiling a latex document in a terminal, the process goes as follows:

% The verbatim environment is for inserting unformatted texts in documents.
% bibtex will whine about how I didn't include the publishers for the books that
% I referenced in my examples. Tell bibtex that I don't care.
\begin{verbatim}
$ pdflatex filename.tex
$ bibtex filename.aux
$ pdflatex filename.tex
$ pdflatex filename.tex
\end{verbatim}

When using a GUI application like TeXmaker, there should be compile buttons for
both (pdf)latex and bibtex. The compilation process will use the same order as
the process in the terminal. First, compile the latex document. Next, use
bibtex. After that, compile the latex document twice and you should be good to
go.

\section{Mathematics typesetting}

There are way too many math symbols to go through here, but there are plenty of
reference tables online. One general rule is that Greek letters are created by
typing out the name. For capital Greek letters, capitalize the name. Example:
$\alpha$, $\beta$, $\gamma$, $\Delta$, $\Psi$, $\psi$, $\chi$. Exponents and
subscripts are done by \^ and \_, respectively. % Don't include the \!
For exponents/subscripts that are more than one character, brackets are
required. Example: $x^2$, $x_0$, $A^i_j$, $e^{-x^2}$, $X^{ab}_{cd}$. Square root
is $\sqrt{x}$, and functions like sine, cosine, and tangent get their own
commands. Example: $\sin(x)$, $\exp(x)$, $\arctan(x)$. You could also use
brackets around the arguments of those functions if you like.

\subsection{In-line math mode}

I'll start this section out by talking about inserting math typesetting into the
middle of sentences, like this: $E = mc^2$. As we will see in a second, not all
of the display options you might want are not going to be available by default.
For example, here are two examples of the syntax for fractions:
$\frac{a}{b + c}$ and $\dfrac{a}{b + c}$. The second version is also equivalent
to this: $\displaystyle \frac{a}{b + c}$. Some math functions have a
'displaystyle' option, whereas other are the same either in an in-sentence math
equation or as an equation in an equation environment, which I will go over
next. The main types of math typesetting where you might need to use the
displaystyle option are fractions, sums, and integrals. There might be more, but
I don't have them in my memory right now, so you will have to RTFM.\\

Another thing to note about the in-line math syntax is that $it\ makes\ all\
letters\ italicized$, although you have to use escape sequences for spaces. This
probably isn't the best way to do things and I will discuss other ways later.
However, for single words, it is probably fine.\\

\subsection{Math environments}

I'm now about to show examples of various math environments. We will start with
the equation environment. For an example that uses a variety of math syntax and
symbols, I will use the Schr\"{o}dinger equation \cite{GriffithsQM:1995}. To get
more examples like the umlaut syntax I just used, search for latex special
characters. This environment gives numbered equations. I can even reference
Equation \eqref{quantum} before I have even wrote it, although the reference
will show up as a couple of question marks on the first compile. Compile it
again to get the equation number.

\begin{equation}
    % Equation typesetting can span multiple lines here.
    i \hbar \frac{\partial \Psi}{\partial t} =
    -\frac{\hbar^2}{2m} \nabla^2 \Psi + V \Psi
    \label{quantum} % Label creates a reference that we can use later
    % To reference an eqation, use either \eqref{labelname} or \ref{labelname}.
\end{equation}

Next, I will demonstrate an equation without numbering. The name of the
environment is called $displaymath$. I will demo it with the Biot-Savart Law
from electromagnetism \cite{GriffithsEM:1999}. Hopefully, the equations that I
use can help you out with learning various math syntax.

\begin{displaymath}
    \vec{B} = \frac{\mu_0}{4\pi} \int \frac{I\vec{dl} \times \hat{r}}{r^2}
\end{displaymath}

Another great resource for learning latex syntax for math is wikipedia. Just
find a page with equations and hit the edit button to view the latex source.\\

\subsection{Arrays and grouped equations}

I'll give off some demos of arrays and grouped equations in latex. I should warn
that I rarely use this feature, but it is worth knowing. This feature can get
complicated fast, so I'll start with just a matrix and work up to matrix
equations.\\

Here is the identity matrix:
\begin{displaymath}
    % To have parentheses that adjust automatically to various sizes, use the
    % commands \left( ... \right). Note that every \left must have a \right.
    \left[ % brackets also work with left-right
    \begin{array}{ccc}
        % ccc means 3 centered columns. use l or r for right/left columns
        1 & 0 & 0 \\
        0 & 1 & 0 \\
        0 & 0 & 1 \\
        % Use the ampersand (&) to separate array items in each row.
        % End each row with a line break (\\).
    \end{array}
    \right]
\end{displaymath}

There are other matrix environments available to use. Here is the identity
matrix with one of these environments:
\begin{displaymath}
    \begin{bmatrix} % You do not have to specify column arrangement here
        1 & 0 & 0 \\
        0 & 1 & 0 \\
        0 & 0 & 1 \\
    \end{bmatrix}
    % Other available *matrix environments:
    % matrix, vmatrix, Vmatrix, bmatrix, Bmatrix, and pmatrix
\end{displaymath}

Here are the Lorentz transformations \cite{Lorentz} for motion in the $x$
direction, in matrix form ($c = 1$).

% It is usually good to break up complex math structures into simpler sections.
% It will also make your document more readable to you.
\begin{displaymath}
    \begin{bmatrix}
        t \prime \\
        x \prime \\
        y \prime \\
        z \prime\\
    \end{bmatrix}
    =
    \begin{bmatrix} % It's not necessary to align the ampersands.
        \gamma        & -\beta \gamma & 0 & 0 \\
        -\beta \gamma & \gamma        & 0 & 0 \\
        0             & 0             & 1 & 0 \\
        0             & 0             & 0 & 1 \\
    \end{bmatrix}
    \begin{bmatrix}
        t \\
        x \\
        y \\
        z \\
    \end{bmatrix}
\end{displaymath}

Here is an example of a system of equations. Note that the column arrangement is
different from the first matrix.

\begin{displaymath}
    \begin{array}{lcr}
        2x + 3y + 4z & = & 0 \\
        3x - 8y - 6z & = & 0 \\
        5x + 7y - 7z & = & 0 \\
    \end{array}
\end{displaymath}

Here is the same thing, but with a bracket on the left. In general, the array
environment is more flexible and customizable than the matrix environment, but
both should work well.

\begin{displaymath}
    \left\{
    \begin{array}{lcr}
        2x + 3y + 4z & = & 0 \\
        3x - 8y - 6z & = & 0 \\
        5x + 7y - 7z & = & 0 \\
    \end{array}
    \right. % Note: the left-right commands don't need to match
    % \right. is for when you don't want a bracket on the right. \left. should
    % also work.
\end{displaymath}

There is also an environment called $cases$, which can be useful for piecewise
equations. Here is an example. Like always, the ampersand simple is used as a
delimiter to distinguish sections from one another.

\begin{displaymath}
    f \left(x\right) =
    \begin{cases}
        x^2, & if x \geq 0 \\ % \geq is for greater than or equal to
        -x,  & if x < 0 \\
    \end{cases}
\end{displaymath}

\section{Figures and tables}

% Note the use of grave and apostrophes in place of double quotes.
You are bound to eventually have a plot in your latex document. Tables might
also be necessary as well. The first sections that I will show are ``floating''
figures. That means that latex will automatically place the figure at the
beginning of a page or wherever it sees fit. I'll later show how to disable this
feature. Tables made using the tabular environment don't float.

\subsection{Figures}

To embed a figure, use the figure environment. It is required that you have the
graphicx package set in your preamble. Here is an example of a figure, using a
plot from the final project of my basic semiconductor circuits class (See Figure
\ref{bsc}). LaTeX will automatically place this figure somewhere. \\

\begin{figure}
    \centering % Makes the figure centered
    % Both relative and absolute paths to the image are allowed. However,
    % absolute paths make your .tex file unportable, which defeats part of the
    % purpose of using latex. To include images from a folder called 'pics', do
    % this: \includegraphics{pics/imagename.png}. NEVER use the parent directory
    % symbol (..). DO NOT DO: \includegraphics{../folder/pic.png}.
    % The square brackets lets you set options. The curly brackets lets you set
    % the plot file. I'll put more examples here.
    \includegraphics[width=0.6\textwidth]{eyes_closed_1000_30.png}
    % Captions are optional. In-line math mode works in captions.
    \caption{Spectrogram of an EEG reading computed using the Short-Time Fourier
        Transform.}
    % Labels are also optional, but they help people know what figure you are
    % referring to. Label's sibling command is \ref. It's also worth nothing
    % that labels might not be correct on the first compile. Just recompile and
    % everything should be fine.
    \label{bsc}
\end{figure}

That's a simple example of a figure and it should be sufficient for most
applications. Next is an example of two side-by-side plots. The $minipage$
environment will be used to achieve this (See Figures \ref{cosine} and
\ref{sine}). Captions and labels can also be placed outside of the minipage
environments (See Figure \ref{sincos}) It might be possible to put captions in
both places, but I have never tried that.\\

\begin{figure}
    \centering % Keep entire structure centered
    % When calling the minipage environment, what needs to get specified is the
    % alignment ([t]) and the width. The alignment used in this example aligns
    % to top of the minipage. To align to the bottom of the minipage, use the
    % letter 'b' in place of 't'. There are other alignment options, but I never
    % use them and I don't have them off the top of my head.
    \begin{minipage}[t]{0.45\textwidth}
        % Put the statements that normally go inside of the figure environment
        % here inside the minipage.
        \centering % center figures inside of the minipage
        \includegraphics[width=\textwidth]{cos.pdf}
        \caption{$\cos(x)$}
        \label{cosine}
    \end{minipage}
    % Adding a space in between the two minipages is not necessary, but it makes
    % things look better, IMO.
    \hspace{0.5cm}
    \begin{minipage}[t]{0.45\textwidth}
        \centering
        \includegraphics[width=\textwidth]{sin.pdf}
        \caption{$\sin(x)$}
        \label{sine}
    \end{minipage}
\end{figure}

% Same thing as above, but with one caption for both images.
\begin{figure}
    \centering
    \begin{minipage}[t]{0.45\textwidth}
        \centering
        \includegraphics[width=\textwidth]{cos.pdf}
    \end{minipage}
    \hspace{0.5cm}
    \begin{minipage}[t]{0.45\textwidth}
        \centering
        \includegraphics[width=\textwidth]{sin.pdf}
    \end{minipage}
    % Caption and label can go here instead.
    \caption{$\cos(x)$ and $\sin(x)$}
    \label{sincos}
\end{figure}

LaTeX supports many image types and you can even embed PDF's in LaTeX documents
as figures. It should be mentioned that the size of am image is one of the
determining factors of the size of the output file. For example, if a 25 MB jpeg
image gets embedded into a LaTeX document, the output document will be over 25
MB. Basically, images should be resized to the absolute minimum size necessary
before being included in LaTeX documents to avoid creating incredibly large
output files. Nobody wants to have to deal with a +100 MB file containing a
bunch of images. Size that stuff down.\\

\begin{figure}
    \centering
    \includegraphics[width=0.6\textwidth]{piecewise.pdf}
    \caption{PDF format}
    \label{pdfformat}
\end{figure}
\begin{figure}
    \centering
    \includegraphics[width=0.6\textwidth]{piecewise.png}
    \caption{PNG format}
    \label{pngformat}
\end{figure}

Another stylistic consideration to make is to use pdf instead of png for plots
when possible. The reason for this is that most of the time, when saving plots
to pdf, the output is lossless. When saving to png format, the output is always
lossy and you can tell the difference by looking at a png plot versus a pdf
plot (See Figures \ref{pdfformat} and \ref{pngformat}). Of course, there are
some exceptions. If a PDF plot is large or LaTeX is having trouble embed it,
then using png instead of pdf is an alright choice to make (See Figure
\ref{bsc}). This convention is purely stylistic.\\

\subsection{Tables}

I'll give an example of a table. Like arrays, one must specify the alignment of
the columns. However, one can also choose to have vertical lines in between
columns in the alignment. The syntax is very similar to arrays and matrices, so
I won't say that much about tables here. \\

% To center the table, put it inside the center environment. \centering will
% throw a compiler error. The center environment is not required for the tabular
% environment.
\begin{center}
% The vertical lines in the column arrangement set vertical lines between
% columns. All of these columns are centered, but a different arrangement could
% easily be chosen.
\begin{tabular}{| c | c | c |}
    \hline % Add a horizontal line in between rows.
    % Like arrays, use an ampersand to separate columns.
    Col. 1 & Col. 2 & Col. 3 \\
    \hline
    0.234  & cheesecake & 234.2 \\
    \hline
    OpenBSD & 3.14 & 2.718 \\
    \hline
    134.1324 & 52345.52 & snowboard \\
    \hline
\end{tabular}
\end{center}

\subsection{Non-floating figures}

Sometimes the figure floating of LaTeX can be annoying and inconvenient, so it
needs to be disabled. Here is how to do that (Figure \ref{arctan}).

\begin{figure}[!htbp]
    \centering
    \includegraphics[width=0.5\textwidth]{arctan.pdf}
    \caption{$\tan^{-1}(x)$}
    \label{arctan}
\end{figure}

\section{Conclusion}

That's all for now. Have fun!

% Time to add in the bibliography for the three citations I made in the paper.
% This command sets the style. In my case, I am choosing to leave the
% bibliogaphy unsorted and in the order that items in reference_list.bib are
% listed.
\bibliographystyle{unsrt}
% This command lets latex know the name of the bibliography. Latex knows to use
% the .bib file even though the extension is missing.
\bibliography{reference_list}{}

% That's all folks!
\end{document}
